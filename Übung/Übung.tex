\documentclass{article}

%\usepackage[utf8]{inputenc}		% LuaTex do not need this
%\usepackage[T1]{fontenc}			% LuaTex do not need this
\usepackage{fontspec}				% LuaTex need this
\usepackage{lmodern}
\usepackage{comment}
\usepackage[
	a4paper,
	top=25mm,
	bottom=25mm,
	left=25mm,
	right=25mm,
%	showframe,
]{geometry}
\usepackage{graphicx}
	\graphicspath{ {Bilder/} }

%%%%%%%%%%%%%%%%%%%%%%%%%%%%%%%%%%%%%%%%%%%%%%%%%%%%%%%%%%%%%%%%%%%%%%%%%%%%%%%%
%
% https://tex.stackexchange.com/questions/82993/how-to-change-the-name-of-document-elements-like-figure-contents-bibliogr
% https://tex.stackexchange.com/questions/186946/changing-the-autoref-name-for-chapter
%
%\usepackage[ngerman]{babel}		% LuaTex do not need this
\usepackage{polyglossia}
	\setdefaultlanguage[spelling=new]{german}
	\addto\captionsgerman{
		\renewcommand{\figurename}{Abbildung}
		\renewcommand{\figureautorefname}{Abbildung}
		\renewcommand{\equationautorefname}{Gleichung}
	}


%%%%%%%%%%%%%%%%%%%%%%%%%%%%%%%%%%%%%%%%%%%%%%%%%%%%%%%%%%%%%%%%%%%%%%%%%%%%%%%%
%
% LATEX Mathematical Symbols
% https://reu.dimacs.rutgers.edu/Symbols.pdf
% https://en.wikibooks.org/wiki/LaTeX/Mathematics
%
\usepackage{amssymb}
\usepackage{amsthm}


%%%%%%%%%%%%%%%%%%%%%%%%%%%%%%%%%%%%%%%%%%%%%%%%%%%%%%%%%%%%%%%%%%%%%%%%%%%%%%%%
%
% TikZ
%
\usepackage{tikz}
	\usetikzlibrary{arrows.meta,graphs,shapes}
	% Georg
	\tikzstyle{vertex}=[circle, minimum size=15pt, draw=black]
	\tikzstyle{selected vertex} = [vertex, fill=red!24]
	\tikzstyle{edge} = [draw,thick,-]
	\tikzstyle{weight} = [font=\small]
	\tikzstyle{selected edge} = [draw,line width=5pt,-,red!50]
	\tikzstyle{ignored edge} = [draw,line width=5pt,-,black!20]
	\pgfdeclarelayer{background}
	\pgfsetlayers{background,main}
	% Albert
	\tikzstyle{node_normal} = [circle, draw, inner sep=0pt, minimum size=20pt]
	\tikzstyle{node_deleted} = [circle, draw, dashed, inner sep=0pt, minimum size=20pt]
	\tikzstyle{node_bold} = [circle, draw, ultra thick, inner sep=0pt, minimum size=20pt]
	\tikzstyle{edge_deleted} = [dashed]
	\tikzstyle{edge_hamilton} = [ultra thick]
	\tikzstyle{edge_normal} = []
	\tikzstyle{edge_with_arrow} = [-{Latex[scale=1.5]}]
\usepackage{tkz-berge}
\usepackage{tkz-graph}
\makeatletter
\pgfmathdeclarefunction{alpha}{1}{%
	\pgfmathint@{#1}%
	\edef\pgfmathresult{\pgffor@alpha{\pgfmathresult}}%
}
%%%%%%%%%%%%%%%%%%%%%%%%%%%%%%%%%%%%%%%%%%%%%%%%%%%%%%%%%%%%%%%%%%%%%%%%%%%%%%%%
%
% https://www.dante.de/events/Archiv/dante2012/Programm/Vortraege/vortrag-ferber.pdf
%
\usepackage[
	breaklinks,
	colorlinks,
	linkcolor=black,
	urlcolor=black,
	citecolor=black,
	pdfencoding=auto,
]{hyperref}
%%%%%%%%%%%%%%%%%%%%%%%%%%%%%%%%%%%%%%%%%%%%%%%%%%%%%%%%%%%%%%%%%%%%%%%%%%%%%%%%
%
%
%
\usepackage{verbatim}
\usepackage{float}
\usepackage{subcaption}
	\captionsetup{subrefformat=parens}


%%%%%%%%%%%%%%%%%%%%%%%%%%%%%%%%%%%%%%%%%%%%%%%%%%%%%%%%%%%%%%%%%%%%%%%%%%%%%%%%
%
%
%
\title{Hamiltonsche Graphen\\ Übung}


\author{
%  Albert Kasdorf\and
%  Andreas Janster\and
%  Alex Bibanaev\and
%  Georg Braun
}
\date{09.05.2018}


%%%%%%%%%%%%%%%%%%%%%%%%%%%%%%%%%%%%%%%%%%%%%%%%%%%%%%%%%%%%%%%%%%%%%%%%%%%%%%%%
%
%
%
\begin{document}
	\maketitle
	\thispagestyle{empty}
	\section*{Zeichnen Sie in den folgenden Graphen einen Hamilton-Kreis ein und überlegen Sie sich, warum die anderen Graphen nicht hamiltonsch sind.}
	\begin{figure}[h]
		\centering
		\begin{subfigure}[b]{0.3\linewidth}
			\centering
			\begin{tikzpicture}
			\def\inc{((2*pi)/5)+pi/2}
			\def\ro{2}
			\def\ri{1}
			
			\node (a) at ({cos(deg(0*\inc))*\ro}, {sin(deg(0*\inc))*\ro}) [node_normal] {a};
			\node (b) at ({cos(deg(1*\inc))*\ro}, {sin(deg(1*\inc))*\ro}) [node_normal] {b};
			\node (c) at ({cos(deg(2*\inc))*\ro}, {sin(deg(2*\inc))*\ro}) [node_normal] {c};
			\node (d) at ({cos(deg(3*\inc))*\ro}, {sin(deg(3*\inc))*\ro}) [node_normal] {d};
			\node (e) at ({cos(deg(4*\inc))*\ro}, {sin(deg(4*\inc))*\ro}) [node_normal] {e};
			
			\node (f) at ({cos(deg(0*\inc))*\ri}, {sin(deg(0*\inc))*\ri}) [node_normal] {f};
			\node (g) at ({cos(deg(1*\inc))*\ri}, {sin(deg(1*\inc))*\ri}) [node_normal] {g};
			\node (h) at ({cos(deg(2*\inc))*\ri}, {sin(deg(2*\inc))*\ri}) [node_normal] {h};
			\node (i) at ({cos(deg(3*\inc))*\ri}, {sin(deg(3*\inc))*\ri}) [node_normal] {i};
			\node (j) at ({cos(deg(4*\inc))*\ri}, {sin(deg(4*\inc))*\ri}) [node_normal] {j};
			
			\draw[edge_normal] (a)--(b)--(c)--(d)--(e)--(a);
			\draw[edge_normal] (f)--(g)--(h)--(i)--(j)--(f);
			\draw[edge_normal] (a)--(f) (b)--(g) (c)--(h) (d)--(i) (e)--(j);
			\end{tikzpicture}
			\caption{}
		\end{subfigure}
		\hfill
		\begin{subfigure}[b]{0.3\linewidth}
			\centering
			\begin{tikzpicture}
			\node (a) at (0,0) [node_normal] {a};
			\node (b) at (0,2) [node_normal] {b};
			\node (c) at (1,3) [node_normal] {c};
			\node (d) at (2,2) [node_normal] {d};
			\node (e) at (2,0) [node_normal] {e};
			\node (f) at (1,1) [node_normal] {f};
			
			\draw[edge_normal] (f)--(a) (f)--(b) (f)--(c) (f)--(d) (f)--(e);
			\end{tikzpicture}
			\caption{}
		\end{subfigure}
		\hfill
		\begin{subfigure}[b]{0.3\linewidth}
			\centering
			\begin{tikzpicture}
			\node (a) at (0,1) [node_normal] {a};
			\node (b) at (1,2) [node_normal] {b};
			\node (c) at (1,0) [node_normal] {c};
			\node (d) at (2,1) [node_normal] {d};
			\node (e) at (3,2) [node_normal] {e};
			\node (f) at (3,0) [node_normal] {f};
			\node (g) at (4,1) [node_normal] {g};
			
			\draw[edge_normal] (a)--(b)--(e)--(g)--(f)--(c)--(a);
			\draw[edge_normal] (b)--(c)--(d)--(e)--(f)--(d)--(b);
			\end{tikzpicture}
			\caption{}
		\end{subfigure}
		\par\bigskip
		\begin{subfigure}[b]{0.3\linewidth}
			\centering
			\begin{tikzpicture}
			\def\inc{((2*pi)/6)+pi/2}
			\def\ro{1.5}
			\node (a) at ({cos(deg(0*\inc))*\ro}, {sin(deg(0*\inc))*\ro}) [node_normal] {a};
			\node (b) at ({cos(deg(1*\inc))*\ro}, {sin(deg(1*\inc))*\ro}) [node_normal] {b};
			\node (c) at ({cos(deg(2*\inc))*\ro}, {sin(deg(2*\inc))*\ro}) [node_normal] {c};
			\node (d) at ({cos(deg(3*\inc))*\ro}, {sin(deg(3*\inc))*\ro}) [node_normal] {d};
			\node (e) at ({cos(deg(4*\inc))*\ro}, {sin(deg(4*\inc))*\ro}) [node_normal] {e};
			\node (f) at ({cos(deg(5*\inc))*\ro}, {sin(deg(5*\inc))*\ro}) [node_normal] {f};
			
			\draw[edge_normal] (a)--(c)--(e)--(a);
			\draw[edge_normal] (b)--(d)--(f)--(b);
			\end{tikzpicture}
			\caption{}
		\end{subfigure}
		\hfill
		\begin{subfigure}[b]{0.3\linewidth}
			\centering
			\begin{tikzpicture}
			\def\inc{((2*pi)/5)+pi/2}
			\def\ro{2}
			\node (A) at ({cos(deg(0*\inc))*\ro}, {sin(deg(0*\inc))*\ro}) [node_normal] {a};
			\node (B) at ({cos(deg(1*\inc))*\ro}, {sin(deg(1*\inc))*\ro}) [node_normal] {b};
			\node (C) at ({cos(deg(2*\inc))*\ro}, {sin(deg(2*\inc))*\ro}) [node_normal] {c};
			\node (D) at ({cos(deg(3*\inc))*\ro}, {sin(deg(3*\inc))*\ro}) [node_normal] {d};
			\node (E) at ({cos(deg(4*\inc))*\ro}, {sin(deg(4*\inc))*\ro}) [node_normal] {e};
			
			\draw[edge_normal] (A)--(B)--(C)--(D)--(E)--(A);
			\draw[edge_normal] (A)--(C)--(E)--(B)--(D)--(A);
			\end{tikzpicture}
			\caption{}
		\end{subfigure}
		\hfill
		\begin{subfigure}[b]{0.3\linewidth}
			\centering
			\begin{tikzpicture}
			\node (a) at (0,2) [node_normal] {a};
			\node (b) at (1,2) [node_normal] {b};
			\node (c) at (2,2) [node_normal] {c};
			\node (d) at (3,2) [node_normal] {d};
			\node (e) at (4,2) [node_normal] {e};
			
			\node (f) at (0,1) [node_normal] {f};
			\node (g) at (1,1) [node_normal] {g};
			\node (h) at (2,1) [node_normal] {h};
			\node (i) at (3,1) [node_normal] {i};
			\node (j) at (4,1) [node_normal] {j};
			
			\node (k) at (0,0) [node_normal] {k};
			\node (l) at (1,0) [node_normal] {l};
			\node (m) at (2,0) [node_normal] {m};
			\node (n) at (3,0) [node_normal] {n};
			\node (o) at (4,0) [node_normal] {o};
			
			\draw[edge_normal] (a)--(b)--(c)--(d)--(e);
			\draw[edge_normal] (f)--(g)--(h)--(i)--(j);
			\draw[edge_normal] (k)--(l)--(m)--(n)--(o);
			\draw[edge_normal] (a)--(f)--(k);
			\draw[edge_normal] (b)--(g)--(l);
			\draw[edge_normal] (c)--(h)--(m);
			\draw[edge_normal] (d)--(i)--(n);
			\draw[edge_normal] (e)--(j)--(o);
			
			\end{tikzpicture}
			\caption{}
		\end{subfigure}
		\par\bigskip
		\begin{subfigure}[b]{0.45\linewidth}
			\centering
			\begin{tikzpicture}
			\node (a) at (0,1) [node_normal] {a};
			\node (b) at (1,2) [node_normal] {b};
			\node (c) at (1,0) [node_normal] {c};
			\node (d) at (2,1) [node_normal] {d};
			\node (e) at (3,2) [node_normal] {e};
			\node (f) at (3,0) [node_normal] {f};
			\node (g) at (4,1) [node_normal] {g};
			
			\draw[edge_normal] (a)--(b)--(d)--(c)--(a);
			\draw[edge_normal] (d)--(e)--(g)--(f)--(d);
			\end{tikzpicture}
			\caption{}
		\end{subfigure}
		\hfill
		\begin{subfigure}[b]{0.45\linewidth}
			\centering
			\begin{tikzpicture}
			\def\inc{((2*pi)/5)+pi/2}
			\def\ro{2}
			\def\ri{1}
			
			\node (A) at ({cos(deg(0*\inc))*\ro}, {sin(deg(0*\inc))*\ro}) [node_normal] {a};
			\node (B) at ({cos(deg(1*\inc))*\ro}, {sin(deg(1*\inc))*\ro}) [node_normal] {b};
			\node (C) at ({cos(deg(2*\inc))*\ro}, {sin(deg(2*\inc))*\ro}) [node_normal] {c};
			\node (D) at ({cos(deg(3*\inc))*\ro}, {sin(deg(3*\inc))*\ro}) [node_normal] {d};
			\node (E) at ({cos(deg(4*\inc))*\ro}, {sin(deg(4*\inc))*\ro}) [node_normal] {e};
			
			\node (F) at ({cos(deg(0*\inc))*\ri}, {sin(deg(0*\inc))*\ri}) [node_normal] {f};
			\node (G) at ({cos(deg(1*\inc))*\ri}, {sin(deg(1*\inc))*\ri}) [node_normal] {g};
			\node (H) at ({cos(deg(2*\inc))*\ri}, {sin(deg(2*\inc))*\ri}) [node_normal] {h};
			\node (I) at ({cos(deg(3*\inc))*\ri}, {sin(deg(3*\inc))*\ri}) [node_normal] {i};
			\node (J) at ({cos(deg(4*\inc))*\ri}, {sin(deg(4*\inc))*\ri}) [node_normal] {j};
			
			\draw[edge_normal] (A)--(B) (B)--(C) (C)--(D) (D)--(E) (E)--(A);
			\draw[edge_normal] (F)--(H) (H)--(J) (J)--(G) (G)--(I) (I)--(F);
			\draw[edge_normal] (A)--(F) (B)--(G) (C)--(H) (D)--(I) (E)--(J);
			\end{tikzpicture}
			\caption{}
		\end{subfigure}
	\end{figure}


%	\section{Entfernen Sie einige Knoten aus den Graphen, was fällt Ihnen bezüglich der entstehenden Zusammenhangskomponenten auf?}
%	\begin{figure}[H]
%		\centering
%		\begin{subfigure}{0.3\linewidth}
%			\centering
%			\begin{tikzpicture}
%			\def\inc{((2*pi)/3)+pi/2}
%			\def\ro{2}
%			\def\ri{0.75}
%			
%			\node (a) at ({cos(deg(0*\inc))*\ro}, {sin(deg(0*\inc))*\ro}) [node_normal] {a};
%			\node (b) at ({cos(deg(1*\inc))*\ro}, {sin(deg(1*\inc))*\ro}) [node_normal] {b};
%			\node (c) at ({cos(deg(2*\inc))*\ro}, {sin(deg(2*\inc))*\ro}) [node_normal] {c};
%			
%			\node (d) at ({cos(deg(0*\inc))*\ri}, {sin(deg(0*\inc))*\ri}) [node_normal] {d};
%			\node (e) at ({cos(deg(1*\inc))*\ri}, {sin(deg(1*\inc))*\ri}) [node_normal] {e};
%			\node (f) at ({cos(deg(2*\inc))*\ri}, {sin(deg(2*\inc))*\ri}) [node_normal] {f};
%
%			\draw[edge_normal] (a)--(b)--(c)--(a);
%			\draw[edge_normal] (d)--(e)--(f)--(d);
%			\draw[edge_normal] (a)--(d) (b)--(e) (c)--(f);
%			\end{tikzpicture}
%		\end{subfigure}
%		\hfill
%		\begin{subfigure}{0.2\linewidth}
%			\centering
%			\begin{tikzpicture}
%			\node (a) at (0,0) [node_normal] {a};
%			\node (b) at (0,2) [node_normal] {b};
%			\node (c) at (1,3) [node_normal] {c};
%			\node (d) at (2,2) [node_normal] {d};
%			\node (e) at (2,0) [node_normal] {e};
%			\node (f) at (1,1) [node_normal] {f};
%			
%			\draw[edge_normal] (a)--(b) (a)--(f) (a)--(e);
%			\draw[edge_normal] (b)--(c) (b)--(d) (b)--(f);
%			\draw[edge_normal] (d)--(c) (d)--(f) (d)--(e);
%			\draw[edge_normal] (e)--(f);
%			\end{tikzpicture}
%		\end{subfigure}
%		\hfill
%		\begin{subfigure}{0.4\linewidth}
%			\centering
%			\begin{tikzpicture}
%			\node (a) at (0,0.5) [node_normal] {a};
%			\node (b) at (1,-0.5) [node_normal] {b};
%			\node (c) at (1,1.5) [node_normal] {c};
%			\node (d) at (2,0.5) [node_normal] {d};
%			\node (e) at (3.5,0) [node_normal] {e};
%			\node (f) at (3,-1) [node_normal] {f};
%			\node (g) at (4,-1) [node_normal] {g};
%			
%			\node (h) at (4,1) [node_normal] {h};
%			\node (i) at (5,2) [node_normal] {i};
%			\node (j) at (3,2) [node_normal] {j};
%			\node (k) at (4,3) [node_normal] {k};
%			
%			\draw[edge_normal] (a)--(b)--(c)--(a);
%			\draw[edge_normal] (c)--(d)--(b);
%			\draw[edge_normal] (e)--(f)--(g)--(e);
%			\draw[edge_normal] (h)--(i)--(k)--(j)--(h);
%			\draw[edge_normal] (i)--(j);
%			\draw[edge_normal] (h)--(d)--(j);
%			\draw[edge_normal] (d)--(e)--(h);
%			\end{tikzpicture}
%		\end{subfigure}
%	\end{figure}

\end{document}
